\documentclass[boxes,serif]{seastaralgebras_expository}
\usepackage{seastaralgebras}

\renewcommand{\sth}{\mid}

\title{Calculus I for Life \& Social Sciences}
\shorttitle{MATH 135}
\author{Karuna Sangam} % Enter your name
% \collaborators{Leonhard Euler, Bernard Bolzano} % Enter your collaborators or comment out
\date{\today} % Date of most recent version
\institution{Rutgers University} % Enter your university affiliation

\begin{document}

\maketitle

% You can comment out code by using the percentage symbol

\section{Algebra Review}

% This is a test.

\subsection{Properties of Arithmetic}

In this calculus course, it is important to be comfortable with the foundations of the mathematics we will be using. Much of this will be familiar to you, but it is valuable nonetheless to remind ourselves of the operations of arithmetic we will encounter.

\begin{defn} \label{def:realnumbers}
    The \emph{real numbers} (denoted $\bbr$), 
\end{defn}

\begin{defn} \label{def:addition}
    Given two real numbers $a$ and $b$
\end{defn}




% \section*{Appendix}
% \begin{proof}[Proof of Properties~\ref{proper:division}]
%     TBD
% \end{proof}


% \begin{thm}
%     This is a theorem.
% \end{thm}
% \begin{proof}
% Your answer here
% \end{proof}

% \begin{example}
%     This is an example.
% \end{example}


% \subsection{Exponents and Radicals}

% \begin{defn}
% This is a definition.
% \end{defn}

% \begin{thm}
%     This is a theorem.
% \end{thm}
% \begin{proof}
% Your answer here
% \end{proof}

% \begin{example}
%     This is an example.
% \end{example}



\end{document}
