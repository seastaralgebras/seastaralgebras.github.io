\documentclass{article}
\usepackage[utf8]{inputenc}
\usepackage{hyperref}

\usepackage{titlesec}

\usepackage{geometry}
\geometry{letterpaper, portrait, margin=.75in}

\titleformat{\section}[block]{\Large\bfseries\filcenter}{}{1em}{}

\title{Math 111 (Precalculus Part I)}
\author{Rutgers, The State University of New Jersey}
\date{Fall 2023}

\begin{document}

\maketitle

\section*{General Information}
% \subsection*{Course}
\begin{center}
    \begin{tabular}{rl}
        \textbf{Course:} & 01:640:111 \\
        \textbf{Sections:} & 40-42 \\
        \textbf{Semester:} & Fall 2023 \\
        \textbf{Lecture Instructor:} & Karuna Sangam \\
        \textbf{Recitation Instructor:} & Eric Wasserman \\
    \end{tabular}
\end{center}

\subsection*{Meeting Times}

\begin{center}
    \begin{tabular}{rl}
        \textbf{Lecture:} & Mon/Wed 2:00--3:20pm (TIL-242, LIV) \\
        \textbf{Recitation (40):} & Tue 3:50--5:10pm (TIL-103A, LIV) \\
        \textbf{Recitation (41):} & Tue 5:40--7:00pm (LSH-A121, LIV) \\
        \textbf{Recitation (42):} & Tue 7:30--8:50pm (LSH-A121, LIV) \\
    \end{tabular}
\end{center}

% \subsection*{Contact}

% \begin{center}
%     \begin{tabular}{rlrl}
%         \textbf{Lecturer:} & Karuna Sangam & \textbf{Recitation Instructor:} & Eric Wasserman \\
%         \textbf{Email:} & \href{mailto:karuna.sangam@rutgers.edu}{karuna.sangam@rutgers.edu} & \textbf{Email:} & \href{mailto:ehw29@math.rutgers.edu}{ehw29@math.rutgers.edu} \\
%         \textbf{Office Hours:} & Mon/Wed 4:30--5:30pm (Hill-622, Busch), & \textbf{Office Hours:} & TBD (TBD) \\
%         & Fri 1--2pm (Zoom) & & \\
%         \textbf{Zoom:} & \href{https://rutgers.zoom.us/my/kss159}{https://rutgers.zoom.us/my/kss159} & & \\
%          & (Password: epsilon) & & 
%     \end{tabular}
% \end{center}

% Additional office hours can be requested by appointment.




\section*{Course Description}
This course provides an essential foundation and skills for students to next take, and succeed in, Pre-Calculus Part II (Math 112). Topics to be addressed include exponents, radicals, factoring, algebraic expressions, algebraic equations, inequalities, graphing, and multiple functions including linear, quadratic, power, root, absolute value and rational. \par
\hfill \par
\noindent \textbf{Course Web Page:} On Math Department website, click \href{https://math.rutgers.edu/academics/undergraduate/courses/927-01-640-111-precalculus-i}{here}. Contains numerous exam preparation materials and syllabus. \par
\hfill \par
\noindent \textbf{Syllabus:} Topic only version may be found on the Math 111 course page. \par
\hfill \par
\noindent \textbf{Prerequisites:} Math 026 (Intermediate Algebra) or demonstrated proficiency in Intermediate Algebra through placement test. \par
\hfill \par
\noindent \textit{The two-course sequence Math 111/112 covers the same material as Math 115 but at a slower pace and begins with an extensive review of intermediate algebra.} \par
\hfill \par
\noindent \textbf{Study Time:} Both concepts and skills are emphasized in this course. In order to develop your skills properly, and thoroughly understand the concepts, most students need to put in 3-4 hours of study time for every hour spent in class. It is also expected that study time would increase beyond the 3:l ratio, one to two weeks before each exam, and two to three weeks before the final. It has been shown that students who have reviewed material on a topic prior to it being covered in lecture have higher success in the course. In that regard, for each topic to be covered, WebAssign has short videos (under Resources) to help you better prepare. 
\par 
\hfill \par
\noindent \textbf{A Note:} This course can be challenging for many, and it is important to stay on top of the material. If you find yourself struggling at any point, it's best to stay proactive and ask for help if you are concerned about your performance in the course. I am here to help you succeed in this course, and I am happy to go over the material with you during office hours or by appointment. I will list some resources (such as tutoring centers) at the end. I also encourage you to study in groups and work on the practice problems, though all work submitted must be your own. While you will all be evaluated individually, mathematics is best done as a team sport. \par
\hfill \par


\subsection*{Important (Tentative) Dates}

\begin{center}
    \begin{tabular}{rl}
        \textbf{Exam 1:} & Wednesday, Sep. 20th \\
        \textbf{Exam 2:} & Monday, Oct. 16th \\
        \textbf{Exam 3:} & Wednesday, Nov. 15th \\
        \textbf{Final Exam:} & Friday, Dec. 15th, from 12--3pm \\
        \textbf{Regular classes end:} & Wednesday, Dec. 13th
    \end{tabular}
\end{center}

\noindent This syllabus is subject to change at any point during the semester.

\section*{Textbooks and Materials}
\noindent \textbf{Text:} \textit{Precalculus: Mathematics for Calculus w/ Enhanced WebAssign}, 8th Edition, Stewart, etc., Cengage Publishing. \par
% \hfill \par
\begin{itemize}
    \item WebAssign software is required for homework. It will be accessed through the class Canvas site.
    \item An e-version of the text will be provided with WebAssign. Hard copy of text is optional.
    \item WebAssign purchased from Barnes \& Noble (First Day Program) is billed to your Rutgers student account.
    \item Those repeating the class may use previously purchased software and opt out of First Day Program.
\end{itemize}
\textit{For more information on WebAssign access and First Day Program opt out, please see the document ``Math~111 WebAssign How To''.} \par
\hfill \par
\noindent \textbf{Technology Requirements:} Students will need internet access, a computer and audio to complete the homework through WebAssign. The same access with a Webcam can also be beneficial to leverage online Zoom meetings where available. A graphing calculator will be required for some quizzes and exams (more information provided later). For the Math department policy (as of Fall 2022), please click \href{https://math.rutgers.edu/academics/undergraduate/1693-technology-requirements-for-math-courses-in-fall-2022}{here}. \par
\hfill \par

\section*{Course Structure and Evaluation}

Final course average will be calculated as follows: \par
\begin{center}
    \begin{tabular}{|l|l|}
        \hline
        Homework & 5\% \\
        \hline
        Quizzes & 15\% \\
        \hline
        Exam 1 & 7\% \\
        \hline
        Exam 2 & 13\% \\
        \hline
        Exam 3 & 20\% \\
        \hline
        Final Exam & 40\% \\
        \hline
        Total & 100\% \\
        \hline
    \end{tabular}
\end{center}
Final course letter grades based on final course average calculated as follows***:
\begin{center}
    \begin{tabular}{|l|l|}
        \hline
        A & 90--100 \\
        \hline
        B+ & 86--89 \\
        \hline
        B & 80--85 \\
        \hline
        C+ & 76--79 \\
        \hline
        C & 70--75 \\
        \hline
        D & 65--69 \\
        \hline
        F & 0--64 \\
        \hline
    \end{tabular}
\end{center}
***One letter grade will be reduced for excessive absence. \par
\hfill \par
\noindent \textbf{Regardless of exam grades during the term, no student recieving below 50\% on the Final Exam will pass the course.} \par
\hfill \par
\noindent \textbf{Course Retake Policy:} Students who have failed Math 111 twice, will not be able to take it a third time at Rutgers. \par
\hfill \par
\noindent \textbf{Homework:} Assigned homework problems enable students to validate and enhance their skills in relation to the topic(s) covered. The \textbf{\textit{timely}} completion of homework is an essential component to succeeding in the class. Homework will be submitted only through WebAssign (three submissions allowed), and reviewed at recitation, in line with the deadline established for each assignment. \par
\hfill \par
\noindent \textbf{Quizzes:} There will be 12 quizzes, of which the best 10 scores count (2 lowest dropped). \par
\hfill \par
\noindent \textbf{Exams:} There will be three mid-term exams, each announced at least one week prior to the date, and a cumulative final exam. The Math 111 course page provides significant review material to help prepare for each exam. Location of the final exam will be determined later and will be posted on the Canvas site. \par
\hfill \par


\section*{Policies}

\noindent \textbf{Attendance and Participation:} You are expected to atend class at the official meeting times. All students are expected to participate actively during lecture and recitation.
\par \hfill \par
\noindent More than four (4) unexcused absences will result in lowering the final course grade by one letter (e.g. C to D).
\par \hfill \par
\noindent \textbf{Course Absence:} Students who have been told to quarantine, or are experiencing symptoms of any transmittable disease, should remain at home and not attend in-person class meetings. In such situations, please contact your course instructor to make alternate arrangements commensurate with your ability to engage in class. \par \hfill \par
\noindent 
Students who anticipate being absent for over one week should also email the Dean of Students (\href{http://deanofstudents.rutgers.edu/}{http://deanofstudents.rutgers.edu/}).
\par \hfill \par
\noindent \textbf{Missing Exams \& Quizzes:} If a student must miss an exam/quiz because they are participating in a Rutgers-approved activity, the student must notify the instructor by email before the exam and provide the proof. Likewise, students are eligible for a make-up exam/quiz if missed due to a religious holiday. 
\par \hfill \par \noindent In any case, the student must follow their instructor's policy regarding credit for midterm exams/quizzes that are missed because of a valid excuse. \par
\hfill \par
\noindent \textbf{Calculators:} Calculators are not allowed on Exam 1 and 2 and all quizes before Exam 3. Graphing calculator required for a few quizzs, Exam 3, and Final. May NOT use TI-89, TI-Nspire, HP~Prime, or any with a QWERTY keyboard. \par
\hfill \par
\noindent \textbf{Accommodations:} Students with disabilities may be eligible for accommodations through the Office of Disability Services). If you are requesting accommodations, you must follow the procedures outlined on the ODS website (\href{https://ods.rutgers.edu/}{https://ods.rutgers.edu/}. Quiz/exam accommodations should be discussed with the class instructor with regard to time and location (e.g., through ODS).
\par \hfill \par \noindent
If you do receive accommodations through ODS, please send me your Letter of Accommodations as soon as you can. Of course, I will accept them at any point in the semester, but it is important to do this early so that you are able to get the most out of this class, as well as help me prepare for them.
\par \hfill \par \noindent
\textbf{Academic Integrity:} A fundamental tenet of all educational institutions is academic honesty; academic work depends upon respect for and acknowledgment of the work and ideas of others. Misrepresenting someone else's work as one's own is a serious offense in any academic setting and it will not be condoned.
Students are responsible for understanding the principles of academic integrity fully and abiding by them in all their work at the University. Check the Academic Integrity policy at \href{https://studentconduct.rutgers.edu/processes/academic-integrity}{https://studentconduct.rutgers.edu/processes/academic-integrity}. Violations of the policy are taken very seriously.

% \newpage

\section*{Resources for Students}

\subsection*{Learning}
\begin{itemize}
    \item \textbf{Office Hours:} Students are strongly encouraged to take advantage of this opportunity to ask questions and to get to know their teachers.
    \item \textbf{Math Help Center:} The Mathematics department offers in-person help on Monday through Friday and on Sundays. More information may be found at \href{https://math.rutgers.edu/academics/undergraduate/1679-math-help-center}{https://math.rutgers.edu/academics/undergraduate/1679-math-help-center}.
    \item \textbf{Rutgers Learning Centers:} The learning centers offer tutoring in many mathematics classes. Please go to \href{https:/rlc.rutgers.edu}{https:/rlc.rutgers.edu} for more information.
\end{itemize}

\subsection*{Office of Disability Services}
Lucy Stone Hall, Suite A145 \\ Livingston Campus \\ 54 Joyce Kilmer Avenue \\ Piscataway NJ 08854 \\ (848)~445-6800 \\ \href{https://ods.rutgers.edu}{https://ods.rutgers.edu}
\par
\hfill \par
\noindent Rutgers University welcomes students with disabilities into all of the University's educational programs. In order to receive consideration for reasonable accommodations, a student with a disability must contact the appropriate disability services office at the campus where they are officially enrolled, participate in an intake interview, and provide documentation. You may read the documentation guidelines for Rutgers here: \\
\href{https://ods.rutgers.edu/students/documentation-guidelines}{https://ods.rutgers.edu/students/documentation-guidelines} \par
\noindent If the documentation supports your request for reasonable accommodations, your campus’s disability services office will provide you with a Letter of Accommodations. Please share this letter with your instructors and discuss the accommodations with them as early in your courses as possible.


\subsection*{Counseling, ADAP, \& Psychiatric Servies (CAPS)}
17 Senior Street \\ New Brunswick, NJ 08901 \\ (848)~932-7884 \\ \href{http://www.rhscaps.rutgers.edu}{http://www.rhscaps.rutgers.edu}
\par
\hfill \par
\noindent CAPS is a University mental health support service that includes counseling, alcohol and other drug assistance, and psychiatric services staffed by a team of professional within Rutgers Health services to support students’ efforts to succeed at Rutgers University. CAPS offers a variety of services that include: individual therapy, group therapy and workshops, crisis intervention, referral to specialists in the community and consultation and collaboration with campus partners.

\subsection*{Student Wellness Services}
The Rutgers “Just In Case” Web App can be found at \href{http://codu.co/cee05e}{http://codu.co/cee05e}. This allows you to access helpful mental health information and resources for yourself or a friend in a mental health crisis on your smartphone or tablet and easily contact CAPS or RUPD.

\subsection*{Violence Prevention \& Victim Assistance (VPVA)}
3 Bartlett Street \\
New Brunswick, NJ 08901 \\
(848)~932-1181 \\
\href{http://vpva.rutgers.edu/}{http://vpva.rutgers.edu/} 
\par
\hfill \par 
\noindent The Office for Violence Prevention and Victim Assistance provides confidential crisis intervention, counseling and advocacy for victims of sexual and relationship violence and stalking to students, staff and faculty. To reach staff during office hours when the university is open or to reach an advocate after hours, call 848-932-1181.


\subsection*{Scarlet Listeners}
(732)~247-5555\\
\href{http://www.scarletlisteners.com/}{http://www.scarletlisteners.com/} \par \hfill \par
\noindent Free and confidential peer counseling and referral hotline, providing a comforting and supportive safe space.



\end{document}

